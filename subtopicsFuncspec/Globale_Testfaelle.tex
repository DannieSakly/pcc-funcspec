\chapter{Globale Testf\"alle}
\section{Erkl\"arung zur Qualit\"atssicherung}
Wir teilen die Testphase in mehrere Teilphasen ein, um sowohl Komponenten für sich als auch ihr zusammenarbeiten separat zu testen. Automatisierte Tests werden das Backend und Frontend testen. Zudem werden manuelle Tests, zur überpr\"ufung der Bedienbarkeit, durchgef\"uhrt.  Das Ziel ist, 80 Prozent des geschriebenen Codes ausf\"uhrlich zu testen.

\subsection{Komponenten-Tests}
In den Komponenten-Tests werden wir unsere drei Komponenten, die \gls{App}, das \gls{Web-Interface} und den \gls{Web-Dienst}, unabhängig voneinander testen.

\subsection{Integration-Tests}
In den Integration-Tests wird die Kommunikation zwischen den Komponenten getestet. In dieser Phase wird die korrekte Implementierung und Funktion der Schnittstellen sichergestellt.

\subsection{System-Tests}
Die Software wird in einer realen Umgebung installiert und dort unter realen Bedingungen von uns getestet.

\section{Testszenarien}
\subsection{Komponententests}

\subsubsection{\gls{App}}
\begin{enumerate}[\bfseries{TK}10]  
\setcounter{enumi}{99}{}

\item \textbf{Erstmaliges starten der \gls{App}} \hfill\\  
Beim ersten Start der App wird die Anmeldeansicht angezeigt.

\item \textbf{Accountanmeldung} \hfill\\  
In der Anmeldeansicht wird der Benutzer aufgefordert, sich anzumelden. Nach erfolgreicher Anmeldung geht die \gls{App} in den Beobachtungsmodus \"uber.

\item \textbf{Men\"u \"offnen} \hfill\\
Die \gls{App} befindet sich im Beobachtungsmodus. Nach Klicken des Men\"ubuttons wird das Men\"u angezeigt.

\item \textbf{Accountabmeldung} \hfill\\  
Die App zeigt das Men\"u. Klickt der Benutzer auf abmelden, werden die Benutzerdaten von der App gelöscht und er gelangt zur Anmeldeansicht.

\item \textbf{Starten der \gls{App} nach Erstanmeldung} \hfill\\  
Bei Start der App wird die Kameraansicht gezeigt und die App befindet sich im Beobachtungsmodus.

\item \textbf{Beobachtungsmodus} \hfill\\
Im Beobachtungsmodus wird der \gls{Ringpuffer} beschrieben.

\item \textbf{Stoppen des Beobachtungsmodus} \hfill\\  
Die \gls{App} befindet sich im Beobachtungsmodus. Wird die App geschlossen, oder die Kameraansicht beendet so wird der Beobachtungsmodus beendet.

\item \textbf{Aufnahmemodus manuell starten} \hfill\\
Die \gls{App} befindet sich im Beobachtungsmodus. Durch doppeltes Dr\"ucken des Bildschirms wird in den Aufnahmemodus gewechselt. 

\item \textbf{Ausl\"osen des \glslink{G-Sensor}{G-Sensors}} \hfill\\  
Die \gls{App} befindet sich im Beobachtungsmodus. Löst der \gls{G-Sensor} des Ger\"ats aus, so wird in den Aufnahmemodus gewechselt.

\item \textbf{Aufnahmemodus} \hfill\\  
Die \gls{App} wechselt gerade in den Aufnahmemodus. Nach 30 Sekunden wird der Inhalt des Ringpuffers verschl\"usselt auf dem internen Speicherbereich abgelegt. Zudem wird das gespeicherte Video nun unter ``Videos'' angezeigt. 

\item \textbf{Testen der Verschl\"usselung} \hfill\\
Es wird die Korrektheit der Ver- bzw. Entschlüsselung der Videos überprüft.

\item \textbf{Ansicht gespeicherter Videos} \hfill\\
Die \gls{App} befindet sich in der Men\"uansicht. Nach dem Klicken des ``gespeicherte Videos'' Feldes werden alle gespeicherten Videos angezeigt.

\item \textbf{Löschen gespeicherter Videos} \hfill\\
Die \gls{App} zeigt die gespeicherten Videos. Nach dem Klicken des Löschen-Symbols, wird ein Bestätigungsdialog  gezeigt. Durch Klicken der Bestätigung wird das Video gelöscht.

\item \textbf{Einstellungen} \hfill\\
Die \gls{App} befindet sich in der Men\"uansicht. Durch Klicken des ``Einstellungen'' Feldes werden die Einstellungen angezeigt.

\end{enumerate}

\subsubsection{\gls{Web-Dienst}}
\begin{enumerate}[\bfseries{TK}10]  
\setcounter{enumi}{199}{}
\item \textbf{\glslink{anonymisieren}{Anonymisierung} des Videos auf dem \gls{Web-Dienst}} \hfill\\  
Der Web-Dienst hat ein zu verarbeitendes Video. Das Video wird zunächst entschlüsselt, \glslink{anonymisieren}{anonymisiert} anschließend gespeichert und mit dem Benutzeraccount verkn\"upft.

\end{enumerate}

\subsubsection{\gls{Web-Interface}}
\begin{enumerate}[\bfseries{TK}10]  
\setcounter{enumi}{299}{}
\item \textbf{Account erstellen} \hfill\\
Beim \"Offnen der Anmeldeseite besteht die Option, einen Account zu erstellen. Nach Eingabe g\"ultiger Benutzerdaten wird ein Account angelegt.

\item \textbf{Anmelden} \hfill\\
Durch das korrekte Eingeben existierender Benutzerdaten auf der Anmeldeseite, wird der Benutzer angemeldet und auf die Liste seiner Videos weitergeleitet.

\item \textbf{Account verwalten} \hfill\\
Die Website zeigt die Accountverwaltung an. Der Benutzer f\"uhrt eine Account\"anderung durch. Beim n\"achsten Anmeldeversuch sind die ge\"anderten Anmeldedaten ung\"ultig.

\end{enumerate}


\subsection{Integration-Tests}
\subsubsection{\gls{App} <-> \gls{Web-Dienst}}
\begin{enumerate}[\bfseries{TI}10]  
\setcounter{enumi}{99}{}

\item \textbf{Anmelden in der \gls{App}} \hfill\\
Die \gls{App} zeigt das Anmeldefenster an. Der Benutzer gibt seine Anmeldedaten ein. Die Daten werden an den \gls{Web-Dienst} geschickt und verifiziert. Ist die Anmeldung erfolgreich, bestätigt der Web-Dienst dies.

\item \textbf{Video hochladen} \hfill\\
Die \gls{App} zeigt die gespeicherten Videos an. Nach Bet\"atigen des ``Hochladen''-Buttons wird das Video auf den \gls{Web-Dienst} hochgeladen, gesichtert und mit dem Benutzeraccount verkn\"upft.
\end{enumerate}

\subsubsection{\gls{Web-Interface} <-> \gls{Web-Dienst}}
\begin{enumerate}[\bfseries{TI}10]  
\setcounter{enumi}{199}{}

\item \textbf{Account erstellen} \hfill\\
Der Benutzer erstellt \"uber das \gls{Web-Interface} einen Account. Durch den Button ``Account erstellen'' wird eine Anfrage an den \gls{Web-Dienst} gesendet und der Account wird auf der Datenbank hinzugef\"ugt.

\item \textbf{Account\"anderung} \hfill\\
Der Benutzer will Anmeldedaten \"andern. Er gibt seine Änderungen ein und best\"atigt. Darauf wird eine Anfrage des \gls{Web-Interface} an den Server gesendet, der die Daten entsprechend ändert.

\item \textbf{Anzeigen hochgeladener Videos} \hfill\\
Der Benutzer gelangt in zu der Liste seiner Videos. Es wird  eine Anfrage an den \gls{Web-Dienst} geschickt, die alle hochgeladenen Videos dieses Benutzers anfordert. Der Web-Dienst antwortet mit den hochgeladenen Videos des Benutzers und die Videos werden in der angezeigt. 

\end{enumerate}

\subsection{Systemtests}
\begin{enumerate}[\bfseries{TS}10]  
\setcounter{enumi}{99}{}

\item \textbf{Systemtest} \hfill\\  
Das System wird von uns durch Testen unter realen Bedingungen, auf Vollständigkeit und Korrektheit der Funktionalität geprüft. Zudem erfolgt eine Überprüfung der Bedienbarkeit. 
\end{enumerate}
