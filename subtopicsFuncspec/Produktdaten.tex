\chapter{Produktdaten}
Zur Verwendung des Produktes werden vom jeweiligen Teil der Software Daten angefordert und somit auch abgelegt, die jedoch haupts\"achlich auf der Datenbank des Servers gespeichert sind.
\section{App-Daten}
\begin{enumerate}[\bfseries{PD}10]
	\setcounter{enumi}{99}
\item Kundendaten \hfill \\
Es werden Nachname, Vorname sowie die E-Mail Adresse des Kunden abglegt.
\item Einstellungen \hfill \\
Es werden Einstellungen der Kamera, sprich die Auflösung der Aufnahme gespeichert.
\item Ringpuffer \hfill \\
Es werden eine Minute an Video Material im Ringpuffer gespeichert.
\item Videodaten \hfill \\
Es werden Videodaten, sowie Ort und Zeit in die Meta-Daten des Videos gespeichert.
\end{enumerate}

\section{Web-Service}
\begin{enumerate}[\bfseries{PK}10]
	\setcounter{enumi}{199}
\item Videodaten \hfill \\
Es werden Temporär Videodaten, sowie Ort und Zeit in den Meta-Daten des Videos gespeichert.
\item Kundendaten \hfill \\
Es werden Nachname, Vorname sowie die E-Mail Adresse des Kunden in einer Datenbank gespeichert. Jeder Kunde bekommt eine einzigartige ID zugewiesen. 
\end{enumerate}

\section{Webinterface Daten}
\begin{enumerate}[\bfseries{PK}10]
	\setcounter{enumi}{299}
\item Daten \hfill \\
Es werden keine Daten auf dem Webinterface gespeichert. Alle Informationen werden mit REST-Anfragen vom Webservice abgerufen.
\end{enumerate}


