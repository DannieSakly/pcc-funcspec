\chapter{Produktdaten}
Bei der Verwendung des Produktes werden vom jeweiligen Teil der Software Daten erhoben und entsprechend auf dem Server oder dem \gls{Smartphone} abgelegt.
\section{App-Daten}
\begin{enumerate}[\bfseries{PD}10]
	\setcounter{enumi}{99}
\item \textbf{Kundendaten} \hfill \\
Es wird die E-Mail Adresse des Kunden abgelegt.
\item \textbf{Einstellungen} \hfill \\
Es werden Einstellungen der Kamera, die aus Auflösung, Bilder pro Sekunde und \gls{Ringpuffer}länge bestehen, gespeichert.
\item \textbf{\gls{Ringpuffer}} \hfill \\
Es wird eine Minute Videomaterial im \gls{Ringpuffer} gespeichert.
\item \textbf{Videodaten} \hfill \\
Neben den aufgezeichneten Videodaten werden Zeit und Auslöseart in die \gls{Metadaten} des Videos gespeichert.
\end{enumerate}

\section{Web-Service}
\begin{enumerate}[\bfseries{PK}10]
	\setcounter{enumi}{199}
\item \textbf{Videodaten} \hfill \\
Es werden neben den Videodaten Zeit und Auslöseart in die \gls{Metadaten} des Videos gespeichert. Die Daten werden nach einer bestimmten Zeit wieder gelöscht.
\item \textbf{Nutzerdaten} \hfill \\
Es werden die \gls{E-Mail}-Adresse und der \gls{Hash-Code} des Passwortes des Nutzers in einer Datenbank gespeichert. Jeder Nutzer bekommt eine einzigartige ID zugewiesen. 
\end{enumerate}

\section{\gls{Web-Interface}}
\begin{enumerate}[\bfseries{PK}10]
	\setcounter{enumi}{299}
\item \textbf{Daten} \hfill \\
Es werden keine Daten auf dem \gls{Web-Interface} gespeichert. Alle Informationen werden mit REST-Anfragen vom \gls{Web-Dienst} abgerufen.
\end{enumerate}


