\chapter{Entwicklungsumgebung}

\section{Entwicklungstools}

\begin{flushleft}
\begin{tabularx} {\textwidth}{|X|X|} \hline
Android IDE & Andoid Studio \\ \hline
Java IDE & IntelliJ IDEA \\ \hline
Projektmanagement & Atlassian JIRA \\ \hline
Textverarbeitung & LaTeX \\ \hline
TeX-Distribution & TexLive \\ \hline
LaTeX Editor & TexMaker \\ \hline
UML Tool & Umlet \\ \hline
Versionskontrolle & Git \\ \hline
\end{tabularx}
\end{flushleft}

\section{verwendete Technologien}

\begin{flushleft}
\begin{tabularx} {\textwidth}{|X|X|} \hline
Programmiersprache (App, Web-Service und Web-Interface) & Java 8 \\ \hline
Web-Framework & Vaadin 7 \\ \hline
Java Servlet und Http Server & Jetty 9.3.14 \\ \hline
Serverkommunikation & RESTful \\ \hline
RESTful-Framework & Jersey 2.24.1 \\ \hline
Datenbank & PostgreSQL \\ \hline
Videobearbeitung & OpenCV \\ \hline
\end{tabularx}
\end{flushleft}

\section{Beschreibung}

\begin{description}
\item \textbf{Android Studio} \hfill \\
Für die Implementierung der Android App wird die offizielle Android-Entwicklungsumgebung Android Studio von Google verwendet.

\item \textbf{IntelliJ IDEA} \hfill \\
IntelliJ IDEA ist eine Java Entwicklungsumgebung, die zusätzlich zu dem üblichen Umfang anderer gängiger IDEs Support für die Entwicklung mit Vaadin, Jetty und Jersey anbietet.

\item \textbf{Atlassian JIRA} \hfill \\
Atlassian JIRA bietet eine Webanwendung zur Projektverwaltung. Dort werden Aufgaben erfasst, verwaltet und dokumentiert.

\item \textbf{LaTeX} \hfill \\
Um eine einfache, einheitliche und stabile Formattierung zu gewährleisten wird zur Texterstellung LaTeX  anstelle klassischer Texteditoren wie Word verwendet. Umgesetzt wird dies durch die Tex-Distribution TexLive und den Editor TexMaker.

\item \textbf{Umlet} \hfill \\
Zum einfachen Entwerfen von UML-Diagrammen wird das Tool Umlet verwendet.

\item \textbf{Git} \hfill \\
Git bietet eine ein teamfähiges (nicht-lineares) Versionskontrollsystem an, über das alle Daten des Projekts erfasst werden.

\item \textbf{Java} \hfill \\
Da alle verwendeten Technologien auf Java basieren, verwenden wir für alle Module (App, Web-Dienst und Web-Interface) Java.

\item \textbf{Vaadin} \hfill \\
Für die Realisierung des Web-Interface wird Vaadin verwendet. Vaadin ermöglicht die Weboberfläche vollständig in Java zu schreiben und bietet moderne responsive Layouts an.

\item \textbf{Jetty} \hfill \\
Für den Web-Service und das Web-Interface läuft auf dem Server Jetty. Jetty bietet eine Kombination aus Java Servlet und Http Server.

\item \textbf{RESTful} \hfill \\
Um Anfragen zwischen den einzelnen Modulen zu vereinheitlichen verwenden wir die Kommunikationsschnittstelle RESTful.

\item \textbf{Jersey} \hfill \\
Für die Umsetzung von RESTful in Java wird das Framework Jersey verwendet.

\item \textbf{PostgreSQL} \hfill \\
Zur Verfaltung der Nutzerdaten und der hochgeladen Videos wird das Datenbanksystem PostgreSQL eingesetzt.

\item \textbf{OpenCV} \hfill \\
Zur Erkennung der zu anonymisierenden Bildbereiche, sowie zur Anwendung der Anonymisierungsfilter werden OpenCV Algorithmen verwendet.

\end{description}

\section{Verschlüsselung}

Um den Datenschutz zu gewährleisten, ist es eine zentrale Funktion der App und des Web-Dienst, dass Videos vor der Anonymisierung ausschließlich verschlüsselt gespeichert werden. Hierbei wendet die App beim Persistieren eine hybride Verschlüsselung an. Dadurch verbindet hybdride Verschlüsselung die Geschwindigkeit von symmetrischer Verschlüsselung mit der Sicherheit asymetrsicher Verschlüsselung.\\
Zunächst wählt die App einen zufälligen symmetrischen Schlüssel. Mit diesem Schlüssel wird das Video symmetrisch verschlüsselt. Anschließend wird der Schlüssel asymmetrisch mit dem öffentlichen Schlüssel des Web-Dienst verschlüsselt. \\
Als kryptographisches Verfahren werden hierbei AES für die symmetrische und RSA für die asymetrische Verschlüsselung verwendet.
