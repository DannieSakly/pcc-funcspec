\chapter{Entwicklungsumgebung}
\section{Beschreibung}
Die Programmiersprache, mit der wir unser Projekt realisieren ist Java. Wir verwenden f\"ur das Web-Interface das Java Web UI Framework Vaddin, welche eine serverseitige Architektur anbietet. Vaddin benutzt das Google Web Toolkit zur Darstellung von Webseiten, mit der wir unsere Website erstellen. \newline 
Eclipse oder Intelij, beides integrierte Entwicklungsumgebungen, werden für die Implementierung genutzt. \newline 
Für die Android Applikation verwenden wir die frei integrierte Entwicklungsumgebung Android Studio, welche sich auf die Intelij IDEA stützt.\newline
F\"ur das Web-Interface nutzen wir einen Webservice namens "XXX", welcher es uns erm\"oglicht, REST-Anfragen zu bearbeiten. \linebreak
Zur Aufgabenverteilung verwenden wir das von Atlassian entwickelte Projektmanagement-Tool JIRA. Dies wird zur Aufgabenverteilung und Dokumentation unserer Arbeit benutzt. 
Um eine Versionskontrolle aller Dokumente zu ermöglichen benutzen wir GIT um nicht-lineare Entwicklung durchzuführen. 
Der Web-Service GitHub ermöglicht uns unsere Repositories privat zu verwalten. \newline
Zum erstellen des Pflichtenhefts und allen anderen Dokumenten verwenden wir die TeX-Distribution TexLive in Kombination mit dem grafischen Editor TexMaker, welcher uns den Latex Code kompiliert und in PDF-Form ausgibt.
\linebreak
Zur Erstellung von UML-Diagrammen wurde das freie unabhängige UML-Werkzeug UMLet verwendet.  Mit diesem wurden Anwendungs- und Aktivitätsdiagramme erstellt, um bestimmte Bedienungen und Abläufe zu veranschaulichen.
\section{Zusammenfassung}